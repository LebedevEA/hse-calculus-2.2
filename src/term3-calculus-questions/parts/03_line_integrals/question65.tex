\Subsection{Билет 65: Гомотопные пути. Односвязные области. Примеры. Существование первообразной относительно отображения.}

\begin{definition}[Гомотопные пути с неподвижными концами]\thmslashn
	
	$\gamma_0, \gamma_1:[a,b]\to \Omega \;\;\; \gamma_0(a) = \gamma_1(a)$ и $\gamma_0(b) = \gamma_1(b)$
	
	Если существует $\gamma:[a, b] \times [0, 1] \to \Omega$ непрерывное в т.ч. $\gamma(t, 0) = \gamma_0(t), \gamma(t, 1) = \gamma_1(t), \;\;\gamma(a, u) = \gamma_0(a)$ и $\gamma(b, u) = \gamma_0(b)$
	
	$\gamma_u(t):= \gamma(t, u)$ -- путь, соединяющий точки $\gamma_0(a)$ и $\gamma_0(b)$
	
\end{definition}

\begin{definition}[Гомотопные замкнутые пути]\thmslashn
	
	$\gamma_0, \gamma_1:[a,b]\to \Omega \;\;\; \gamma_0(a) = \gamma_1(a)$ и $\gamma_0(b) = \gamma_1(b)$
	
	Если существует $\gamma:[a, b] \times [0, 1] \to \Omega$ непрерывна и $\gamma(t, 0) = \gamma_0(t), \gamma(t, 1) = \gamma_1(t), \;\;\gamma(a, u) = \gamma(b, u)$
	
\end{definition}

\begin{remark_author}
	Гомотопные замкнутые пути могут не пересекаться
\end{remark_author}

Несложно понять, что это отношение эквивалентности.

\begin{definition}
	Стягиваемый путь -- замкнутый путь, гомотопный точке.
\end{definition}

\begin{definition}
	Односвязная область -- область, в которой любой замкнутый путь стягиваемый.
\end{definition}

\begin{example}\thmslashn
	
	\begin{enumerate}
		\item 
		Выпуклая область односвязна 
		
		(Выпукла область -- любой отрезок, соединяющий две точки множества, лежит в области целиком)
		
		\item
		Звездная область односвязна
		
		(Звездная область, относительно точки $O$ -- любой отрезок, соединяющий $O$ и точку множества, лежит в области целиком)
		
		\begin{proof}\thmslashn
			
			Пусть выделенная точка -- $0$
			
			$\gamma_1:[a, b] \to \Omega$ -- замкнутая кривая
			
			$\gamma(t, u) = u\gamma_1(t):[a, b]\times[0,1]\to \Omega$
			
		\end{proof}
		
		\item
		$\R^2 \ \{(0, 0)\} $ не является односвязной
		
	\end{enumerate}
	
\end{example}

\begin{exerc}\thmslashn
	
	$\Omega$ -- односвязна, $f:\mathbf{T} \to \Omega$ непрерывная
	
	$\mathbf{T} = \{x^2 + y^2 = 1\}$
	
	$\bar{\mathbf{D}} = \{x^2 + y^2 \leqslant 1\}$
	
	Тогда $f$ можно продолжить до $f:\bar{\mathbf{D}\to \Omega}$ с сохранением непрерывности
	
\end{exerc}