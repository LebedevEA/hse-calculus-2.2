\Subsection{Билет 65: Гомотопные пути. Односвязные области. Примеры. Существование первообразной относительно отображения.}

\begin{definition}[Гомотопные пути с неподвижными концами]\thmslashn
	
	$\gamma_0, \gamma_1:[a,b]\to \Omega \;\;\; \gamma_0(a) = \gamma_1(a)$ и $\gamma_0(b) = \gamma_1(b)$
	
	Если существует $\gamma:[a, b] \times [0, 1] \to \Omega$ непрерывное и
	
	$\gamma(t, 0) = \gamma_0(t), \gamma(t, 1) = \gamma_1(t), \;\;\gamma(a, u) = \gamma_0(a)$ и $\gamma(b, u) = \gamma_0(b)$
	
	$\gamma_u(t):= \gamma(t, u)$ -- путь, соединяющий точки $\gamma_0(a)$ и $\gamma_0(b)$
	
\end{definition}

\begin{definition}[Гомотопные замкнутые пути]\thmslashn
	
	$\gamma_0, \gamma_1:[a,b]\to \Omega \;\;\; \gamma_0(a) = \gamma_0(b)$ и $\gamma_1(a) = \gamma_1(b)$
	
	Если существует $\gamma:[a, b] \times [0, 1] \to \Omega$ непрерывна и
	
	$\gamma(t, 0) = \gamma_0(t), \gamma(t, 1) = \gamma_1(t), \;\;\gamma(a, u) = \gamma(b, u)$
	
\end{definition}

\begin{remark_author}
	Гомотопные замкнутые пути могут не пересекаться
\end{remark_author}

Несложно понять, что гомотопия -- отношение эквивалентности.

\begin{definition}
	Стягиваемый путь -- замкнутый путь, гомотопный точке.
\end{definition}

\begin{definition}
	Односвязная область -- область, в которой любой замкнутый путь стягиваемый.
\end{definition}

\begin{example}\thmslashn
	
	\begin{enumerate}
		\item 
		Выпуклая область односвязна 
		
		(Выпукла область -- любой отрезок, соединяющий две точки множества, лежит в области целиком)
		
		\item
		Звездная область односвязна
		
		(Звездная область, относительно точки $O$ -- любой отрезок, соединяющий $O$ и точку множества, лежит в области целиком)
		
		\begin{proof}\thmslashn
			
			Пусть выделенная точка -- $0$
			
			$\gamma_1:[a, b] \to \Omega$ -- замкнутая кривая
			
			$\gamma(t, u) = u\gamma_1(t):[a, b]\times[0,1]\to \Omega$
			
		\end{proof}
		
		\item
		$\R^2 \ \{(0, 0)\} $ не является односвязной
		
	\end{enumerate}
	
\end{example}

\begin{exerc}\thmslashn
	
	$\Omega$ -- односвязна, $f:\mathbb{T} \to \Omega$ непрерывная
	
	$\mathbb{T} = \{x^2 + y^2 = 1\}$
	
	$\bar{\mathbb{D}} = \{x^2 + y^2 \leqslant 1\}$
	
	Тогда $f$ можно продолжить до $f:\bar{\mathbb{D}}\to \Omega$ с сохранением непрерывности
	
\end{exerc}

\begin{definition}\thmslashn
	
	$\gamma:[a,b] \times[c, d] \to \Omega$ непрерывно, $\omega$ локально точная форма
	
	$f:[a, b]\times[c, d] \to \R$ первообразная $\omega$ относительно $\gamma$, если $f$ непрерывна и $\forall$ точки $(\tau, \rho)$ в окрестности $\gamma(\tau, \rho)$  существует такая первообразная $F$, что $f(t, u) = F(\gamma(t, u))$ при $(t, u)$ близких к $(\tau, \rho)$
	
\end{definition}

\begin{remark}\thmslashn
	
	При фиксированном $u = u_0\;\; f(t, u_0)$ -- перообразная вдоль пути $\gamma(t, u_0)$ 

\end{remark}

\begin{theorem}\thmslashn

	Первообразная относительно отображения существует и единствена с точностью до константы

\end{theorem}

\begin{proof}\thmslashn
	
	Покроем $\gamma([a, b]\times[c, d])$такими шариками, что в каждом есть первообразная $\omega$
	
	Воьмум $r > 0$ из леммы Лебега (найдётся $r > 0$ что любое подмножество диаиметра $r$ целиком содержится в каком-то элементе покрытия)
	
	$\gamma$ задана на компакте и непрерывна $\Rightarrow$ равномерно непрерывна на $[a, b]\times[c, d] \Rightarrow$ мы можем выбрать $\varepsilon$ т.ч. растояние между образами точек $< r$
	
	Нарежем прямоугольник на маленькие кусочки диаметра $< \varepsilon$ 
		
	$\gamma([t_{i-1}, t_i]\times[u_{j-1}, u_j])$ целиком попадает в $U_{ij}$, где есть первообразная $F_{ij}$
	
	$f$ на $[t_{0}, t_1]\times[u_{0}, u_1] \;\; f(t, u) = F_{1, 1}(\gamma(t, u))$ 
	
	$f$ на $[t_{1}, t_2]\times[u_{1}, u_2] \;\; f(t, u) = F_{1, 1}(\gamma(t, u))$ 
	
	$U_{11} \cap U_{21} \supset {t_1} \times[u_0, u-1] \Rightarrow F_{11}$ и $F_{21}$ первообразные в $U_{11} \cap U_{21} \Rightarrow$ они отличаются на константу и мы можем подправить $F_{21}$ $\Rightarrow$ мы можем построить первообразную для всей строчки.
	
	Аналогично построим $f_j$ на $[a, b]\times[u_{j-1}, u_j]$
	
	$f_j(t, u_j)$ и $f_{j+1}(t, u_j)$ -- первообразная вдоль пути $\gamma(t, u_j) \Rightarrow$ отличаются на константу
	
	Подправим $f_{j+1}$ так, чтобы константа была равна 0 $\Rightarrow$ мы соединили две строки, везде все хорошо, кроме места склейки.
	
	Проверим в $(t, u_j)$. В окрестности $\gamma(t, u_j)$ есть первообразная $F$, т.ч. $f_j(t, u) = F(\gamma(t, u))$ и первообразная $\tilde{F}$, т.ч. $f_{j+1}(t, u) = \tilde{F}(\gamma(t, u))$, но обе это первообразные равны, тк при $u = u_j$ левые части равны
	
\end{proof}
