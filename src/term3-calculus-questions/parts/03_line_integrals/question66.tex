\Subsection{Билет 66: Две теоремы об интегралах от локально точной формы по гомотопным путям. Точность форм в односвязных областях.}

\begin{theorem}\thmslashn
	
	$\gamma_0$ и $\gamma_1$ гомотопные пути с неподвижными концами 
	
	$\omega$ -- локально точная форма
	
	Тогда $\int\limits_{\gamma_0} \omega = \int\limits_{\gamma_1} \omega$
	
\end{theorem}

\begin{proof}\thmslashn
	
	$\gamma$ -- гомотопия между $\gamma_0$ и $\gamma_1\;\;\; \gamma:[a, b]\times[0, 1] \to \Omega$
	
	$f$ -- первообразная $\omega$ относительно гомотопии $\gamma$ 
	
	$\int\limits_{\gamma_0} \omega = f(b, 0) - f(a, 0) \quad \int\limits_{\gamma_1} \omega = f(b, 1) - f(a, 1)$
	
	Докажем, что $f(a, u)$ локально постоянна
	
	Рассмотрим точку $(a, u)$. Существует окрестность точки $\gamma(a, u)$ и первообразная $F$ в ней, т.ч. $f(t, v) = F(\gamma(t, v))$ при $(t, v)$ близки к $(a, u) \Rightarrow$
	
	$f(a, v) = F(\gamma(a, v))$ при $v$ близких к $u$ $\Rightarrow f(a, v) = F(\gamma_0(a))$ при $v$ близких к $u \Rightarrow$ 
	
	$f(a, u)$ -- константа, аналогично $f(b, u)$ -- константа $\Rightarrow$ $f(b, 0) - f(a, 0) = f(b, 1) - f(a, 1)$
	
\end{proof}

\begin{theorem}\thmslashn
	
	$\gamma_0$ стягиваемый путь
	
	$\omega$ -- локально точная форма
	
	Тогда $\int\limits_{\gamma_0} \omega = 0$
	
\end{theorem}

\begin{proof}\thmslashn
	
	$\gamma$ -- гомотопия между $\gamma : [a, b]\times [0, 1] \to \Omega$
	
	$f$ -- первообразная относительно $\gamma$ 
	
	$\int\limits_{\gamma_0} \omega = f(b, 0) - f(a, 0) \quad 0 =  \int\limits_{\gamma_1} \omega = f(b, 1) - f(a, 1)$
	
	Докажем, что $f(b, u) - f(a, u)$ локально постоянна
	
	Возьмем $(a, u)$. В окрестности $\gamma(a, u) = \gamma(b, u)$ существует первообразная $F$, т.ч. $f(t, v) = F(\gamma(t, v))$ при $(t, v)$ близки к $(a, u) \Rightarrow f(a,v) = F(\gamma(a, v))$ при $v$ близких к $u$ 
	
    Возьмем $(b, u)$. В окрестности $\gamma(b, u) = \gamma(a, u)$ существует первообразная $\tilde{F}$, т.ч. $f(t, v) = \tilde{F}(\gamma(t, v))$ при $(t, v)$ близки к $(b, u) \Rightarrow f(b,v) = \tilde{F}(\gamma(b, v))$ при $v$ близких к $u$ ($\gamma(b, v) = \gamma(a, v) \Rightarrow \tilde{F}(\gamma(b, v)) = \tilde{F}(\gamma{a,v})$)

    Ну получили $f(a, v) = F(\gamma(a, v)), f(b, v) = \tilde{F}(\gamma(b, v)) = \tilde{F}(\gamma(a, v))$, а первообразные на константу отличаются
	
	$\Rightarrow f(b, v) - f(a, v)$ постоянна при $v$ близких к $u$
	
\end{proof}

\begin{theorem}\thmslashn
	
	$\Omega$ -- односвязная область, $\omega$ -- локально точная форма $\Rightarrow$ $\omega$ -- точная форма	
\end{theorem}


\begin{proof}\thmslashn
	
    Был критерий, что есть первообразная, есть интеграл по любому замкнутому пути равен 0. Если область односвязная, то все пути стягиваемые (по определению), тогда по предыдущей теореме все интегралы по замкнутым путям равны 0. Значит, есть глобальная первообразная.
\end{proof}