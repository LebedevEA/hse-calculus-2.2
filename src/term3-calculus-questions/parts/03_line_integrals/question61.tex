\Subsection{Билет 61: Первообразная формы. Аналог формулы
	Ньютона–Лейбница. Лемма о существовании ломаной,
	соединяющей точки области. Необходимые и достаточные условия существования первообразной.}

\begin{definition}\thmslashn
	
	$\Omega \subset \R^n$ -- область, если $\Omega$ -- открытое и линейно связное (любые две точки можем соединить) 
	
\end{definition}


\begin{definition}\thmslashn
	
	Первообразная дифф. формы $\omega$ в области $\Omega$ -- функция $F: \Omega \to \R$, т.ч. $dF = \omega$
	
\end{definition}

Напоминание: $dF = \frac{\partial F}{\partial x_1} \cdot dx_1 + \frac{\partial F}{\partial x_2} \cdot dx_2 + \ldots + \frac{\partial F}{\partial x_n} \cdot dx_n = \omega =  f_1 dx_1 + f_2dx_2 + \ldots + f_ndx_n$

Т.е. $\frac{\partial F}{\partial x_k} = f_k$

\begin{theorem}[Аналог формулы Ньютона–Лейбница]\thmslashn
	
	$F$ -- перообразная $\omega$ в $\Omega$, $\gamma$ -- кривая в $\Omega$, соединяющая точки $a$ и $b$. Тогда $\int\limits_{\gamma}\omega = F(b) - F(a)$
	
\end{theorem}

\begin{proof}\thmslashn
	
	$\gamma:[\alpha, \beta]\to\Omega\;\; \gamma(\alpha) = a, \;\gamma(\beta) = b$
	
	$\int\limits_\gamma \omega := \int\limits_\alpha^\beta  \sum\limits_{k=1}^n f_k(\gamma(t)) \cdot \gamma_k'(t)\,dt = \int\limits_\alpha^\beta  \sum\limits_{k=1}^n \frac{\partial F}{\partial x_k} (\gamma(t)) \cdot \gamma_k'(t)\,dt = \int\limits_\alpha^\beta \left( F\circ \gamma \right)'(t)\,dt = $
	
	$= F\circ \gamma \Big|_\alpha^\beta = F(\gamma(\beta)) - F(\gamma(\alpha)) = F(b) - F(a)$
	
\end{proof}

\begin{consequence}\thmslashn

	\begin{enumerate}
		
		\item
		Если у $\omega$ есть первообразная, то интеграл зависит только от концов пути
		
		\item
		Первообразные отличаются друг от друга на константу
		
		\begin{proof}\thmslashn
			
			$F_1$ и $F_2$ отличаются друг от друга на константу
			
			$\int\limits_{\gamma} \omega = F_1(b) - F_1(a) = F_2(b) - F_2(a) \Rightarrow F_2(b) = F_1(b) + (F_2(a) - F_1(a))$
			
		\end{proof}
	
	\end{enumerate}

\end{consequence}

\begin{lemma}
	
	$\Omega$ -- область. Между любыми двумя точками из $\Omega$ найдется ломаная из $\Omega$ со звеньями, параллельными осям координат
	
\end{lemma}

\begin{proof}\thmslashn
	
	$a, b \in \Omega$ возьмем путь, соединяющий эти точки
	
    В каждой точке пути берем шарик, целиком содержащийся в $\Omega$. Выберем конечное покрытие. (множество компактно (путь - непрерывный отрезка, отрезок компактен), шарики покрывают путь и они открыты)
	
	Шарики мы можем упорядочить, тк у каждого шарика есть центр, расположенный на пути $\Rightarrow$ можем упорядочить. Теперь в каждом шарике строим ломанные от точки, входящей в пересечение с предыдущим шариком, до центра шарика и от центра, до точки из пересечения со следующим шариком. За конечное число шагов мы построим ломаную 
	
\end{proof}

\begin{theorem}\thmslashn

	$\Omega$ -- область, $f_1, f_2, \ldots, f_n : \Omega\to \R$ непрерывны
	
	Следущие условия равносильны 
	
	\begin{enumerate}
		\item 
		$\omega =  f_1 dx_1 + f_2dx_2 + \ldots + f_ndx_n$ имеет первообразную в $\Omega$
		
		\item
		$\int\limits_\gamma \omega = 0$ для любого замкнутого контура
		
		\item
		$\int\limits_\gamma \omega = 0$ для любой замкнутой ломаной со звеньями, параллельными осям координат
		
	\end{enumerate} 

\end{theorem}

\begin{proof}\thmslashn
	
	\begin{enumerate}
		\item[1)$\Rightarrow$2)] 
		$F$ -- первообразная, $\gamma$ -- замкнутая кривая, проходящая через точку $a$, то $\int\limits_\gamma \omega = F(a) - F(a) = 0$
		
		\item[2)$\Rightarrow$3)]
		Это частный случай
		
		
		\item[3)$\Rightarrow$1)]
		Зафиксируем точку $a \in \Omega$ и $F(x) := \int\limits_\gamma \omega$, где $\gamma$ -- ломанная со звеньями, параллельными осям координат, соединяющая точки $a$ и $x$. Проверим корректность определения
		
		Возьмем 2 ломанные от точки $a$ до $x$ $\gamma_1$ и $\gamma_2$. Если мы пройдем по $\gamma_1$, а затем вернемся по $\gamma_2$, то это будет замкнутая ломаная $\Rightarrow$ интеграл равен 0. $\Rightarrow$ также $\int\limits_{\gamma_1} \omega = \int\limits_{\gamma_2} \omega$. Значит значение не зависит от прямой, по которой мы двигаемся
		
		Проверим, что $F$ -- перообразная. Докажем, что $\frac{\partial F}{\partial x_i} = f_i$
		
		Чтобы найти производную по первой координате нам надо сдвинуться только по первой координате от $x_1$ до $x_1 + h$ и разделить на перемещение.
		
		$\frac{\partial F}{\partial x_1} = \lim\limits_{h\to 0} \frac{F(x_1+h, x_2, \ldots, x_n) - F(x_1, x_2, \ldots, x_n)}{h} $
		
		Проведем ломаную из $a$ так, чтобы путь от $a$ до $x_1 + h$ проходил через $x_1$ и между $x_1$и $x_1 + h$ был отрезок $I$. Тогда мы сможем посмотреть на интеграл по этому отрезку
		
		\[\lim\limits_{h\to 0} \frac{F(x_1+h, x_2, \ldots, x_n) - F(x_1, x_2, \ldots, x_n)}{h} = \lim\limits_{h\to 0} \frac{1}{h} \int\limits_I \omega = 
        \lim\limits_{h\to 0} \frac{1}{h} \int\limits_0^h f_1(x_1+t, x_2, \ldots, x_n) \,dt = \]\\\[ =
        \lim\limits_{h\to 0} \frac{1}{h} h \cdot f_1(x_1 + \theta \cdot t, x_2, \ldots, x_n) = 
    f_1(x_1, x_2, \ldots, x_n)\]


        Пояснения к переходам:

    В $\omega'$ все частные производные равны 0, кроме $\omega_1' = 1$, так что в определении интеграла 2 рода получаем \[\int_0^h\, dt\, \sum_{i=1}^n f_1(x_1+t, x_2, \ldots, x_n) \cdot 1 + f_2(x_1+t, x_2, \ldots, x_n) \cdot 0 + \ldots = \int_0^h\, dh\, f_1(x_1 + t, x_2, \ldots, x_n)\]

        Предпоследний переход (от интеграла к $\theta$): значение интеграла по отрезку - это значение функции в промежуточной точке, умноженное на длину
		
	\end{enumerate}

\end{proof}