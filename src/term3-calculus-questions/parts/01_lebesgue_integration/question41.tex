\Subsection{Билет 41: Диффеоморфизм. Лемма о «расщеплении» диффеоморфизма. Теорема Линделёфа.}

\begin{definition}\thmslashn 
	
	$\Omega_1$ и $\Omega_2$ -- открытые подмножества в $\R^m$
	
	$\Phi: \Omega_1\to\Omega_2$ -- диффеоморфизм, если $\Phi$ -- биекция, $\Phi \in C^1(\Omega_1)$ и $\Phi^{-1} \in C^1{\Omega_2}$
	
\end{definition}

\begin{theorem}[Линделефа]\thmslashn
	
	$\Omega$ -- открытое множество в $\R^m$
	
	Из любого его покрытия открытыми множествами можно выбрать нбсч подпокрытие
	
\end{theorem}

\begin{proof}\thmslashn
	
	$a\in \Omega$ она покрыта каким-то открытым $U \Rightarrow \exists r_a > 0$, т.ч. $B_{r_{a}}(a) \in U$. Подправим этот шарик так, что его центр будет с рац. координатами и его радиус будет рац. Назовем его $B_a$, $B_a \subset U$ Различных шариков нбчс. 
	
\end{proof}

\begin{lemma}\thmslashn
	
	$\Phi: \Omega\to\tilde{\Omega}$ -- диффеоморфизм $\quad \Omega, \tilde{\Omega} \subset \R^n$
	
	$a\in \Omega$, тогда существует окрестность $U_a$ точки $a$, тч 
	
	$\left.\Phi\right|_{U_a} = \Phi_1 \circ \Phi_2$, где $\Phi_2:U_a \to \Phi_2(U_a)$ и $\Phi_1:\Phi_2(U_a)\to\Phi(U_a)$ -- диффеоморфизмы
	
	причем $\Phi_1$ оставляет на месте $0 < m < n$ координат, а $\Phi_2$ оставляет на месте $n-m$ координат
	
\end{lemma}

\begin{proof}\thmslashn

	$x, u \in \R^m, y, v \in R^{n-m} \quad \Phi(x, y) = (\phi(x, y), \psi(x, y))$
	
	$\Phi_1(u, v) = (u, f(u, v))$ и $\Phi_2(u, v) = (g(u, v), v)$
	
	$\Phi_1 \circ \Phi_2 (x, y) = \Phi_1(g(x,y), y) = (g(x, y), f(g(x, y), y)) = \Phi (x, y) = (\phi(x, y), \psi(x, y)) \Rightarrow$
	
	$\Rightarrow g(x, y) = \phi(x, y)$ и $f(\phi(x, y), y) = \psi(x, y) \Rightarrow f(u, v) = \psi(\Phi_2^{-1}(u, v)) \Rightarrow$ 
	
	$\Rightarrow$ нужна локальная обратимость $\Phi_2$. Она будет, если $\det \Phi_2'(a) \not = 0$
	
	$\det\Phi_2' = \begin{pmatrix}
	g_u' & g_v'\\
	0 & I
	\end{pmatrix} = \det g_u' = \det \phi_u'$
	
	Т.е. нам нужен ненулевой минор $m\times m$ матрицы $\Phi'$, который $\not = 0$. 
	
	Но если все миноры $=0$, то $\det\Phi' = 0$, но это не так.
	
	Мы сейчас могли переставить как-то столбцы и строчки, пока искали ненулевой минор.

\end{proof}


\begin{consequence}\thmslashn

	В условии леммы найдется такая $U_a$, что
	
	$\left. \Phi\right|_{U_a} = \Phi_1\circ \Phi_2 \circ \ldots \circ \Phi_n$, где $\Phi_k$ -- диффеоморфизмы и они оставляют на месте все координаты, кроме одной.

\end{consequence}

\begin{proof}Индукция
\end{proof}

