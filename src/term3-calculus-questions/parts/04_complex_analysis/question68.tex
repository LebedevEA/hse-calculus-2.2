\Subsection{Билет 68: Частные производные $\frac{\partial}{\partial z}$ и $\frac{\partial}{\partial \bar{z}}$. Условия Коши-Римана. Функции с постоянной вещественной частью.}


\begin{designations}
	$\frac{\partial}{\partial z}:= \frac{1}{2}\left( \frac{\partial}{\partial x} - i\frac{\partial}{\partial y}\right) \quad \frac{\partial}{\partial \bar{z}}:= \frac{1}{2}\left( \frac{\partial}{\partial x} + i\frac{\partial}{\partial y}\right)$
\end{designations}

\begin{remark}\thmslashn
	
	Если $f$ голоморфизм в точке $z_0$, то $\frac{\partial f}{\partial z}(z_0) = f'(z_0) \quad \frac{\partial f}{\partial \bar{z}} = 0$
	
\end{remark}

Мотивация: 

Мы дифференциал можем разложить в базис 2 способами. 

$d_a f = \frac{\partial f}{\partial x}(a) \cdot \,dx + \frac{\partial f}{\partial y} (a) \cdot \,dy$

$d_a f = \frac{\partial f}{\partial z}(a) \cdot \,dz + \frac{\partial f}{\partial \bar{z}} (a) \cdot \,d\bar{z}$

Тогда коэффициенты во втором разложении это те формулы, из обозначений.

\begin{theorem}[условия Коши-Римана]\thmslashn
	
	$f:\Omega \to \CC\;\; a\in \Omega \;\; f = g + ih$
	
	$f$ -- дифференцируема в точке $a$, как функция $\R^2 \to \R^2$. 
	
	Тогда следующие условия равносильны
	
	\begin{enumerate}
		\item 
		$f$ -- голомофна в точке $a$
		
		\item
		$d_af$ -- комплексо линейный 
		
		$d_a = \begin{pmatrix}
		\alpha & -\beta\\
		\beta & \alpha
		\end{pmatrix}$
		
		\item
		$\frac{\partial f}{\partial \bar{z}} (a) = 0$
		
		\item
		$\frac{\partial g}{\partial x} = \frac{\partial h}{\partial y}$ и $\frac{\partial g}{\partial y} = - \frac{\partial h}{\partial x}$ (условие Коши-Римана)
	\end{enumerate}
	
\end{theorem}

\begin{proof}\thmslashn
	
	\begin{enumerate}
		\item [1$\Leftrightarrow$ 2]
		Это замечание 2;
		
		\item [2 $\Leftrightarrow$ 4]
		Коэффициенты из матрицы $A$ из замечания 1
		
		\item[3 $\Leftrightarrow$ 4]
		$2\frac{\partial f}{\partial \bar{z}} (a) =  \frac{\partial f}{\partial x} + i\frac{\partial f}{\partial y} =  \frac{\partial g}{\partial x} + i\frac{\partial g}{\partial y} + i\frac{\partial h}{\partial x} - \frac{\partial h}{\partial y}$
		
	\end{enumerate}
	
\end{proof}

\begin{consequence}\thmslashn
	
	$f:\Omega \to \CC$ голоморфна во всех точках и $\Re f =$ const Тогда $f = $ const
	
\end{consequence}

\begin{proof}\thmslashn
	
	$f = g  + ih \quad \frac{\partial g}{\partial x} = \frac{\partial h}{\partial y}$ и $\frac{\partial g}{\partial y} = - \frac{\partial h}{\partial x}$
	
	$g =$ const $ \Rightarrow \frac{\partial h}{\partial y} = 0$ и $ \frac{\partial h}{\partial x} = 0 \Rightarrow h = $ const $\Rightarrow f = $ const
\end{proof}
