\Subsection{Билет 45: Произведение компактов. Непрерывность собственных интегралов с параметром (на компакте и на открытом множестве).}


\begin{lemma}
	
	Декартово произведение компактов -- компакт. 
	
\end{lemma}

\begin{remark}
	
	Метрика на декартовом произведении 
	
	$(X, \rho_X), (Y, \rho_Y)$ -- метрические про-ва $(X\times Y, \rho) \quad \rho((x, y), (\bar{x}, \bar{y})) = \rho_X(x, \bar{x}) + \rho_Y(y, \bar{y})$ -- метрика в $X\times Y$	
	
\end{remark}

\begin{proof}\thmslashn
	
	
	$A\subset X$ -- компакт $\Rightarrow$ возьмём $\varepsilon > 0 \quad A_\varepsilon$ -- конечная $\varepsilon$-сеть для $A$
	
	Аналогично для $B\subset Y$ $B_\varepsilon$ -- конечная $\varepsilon$-сеть для $B$
	
	$A_\varepsilon \times B_\varepsilon$ -- $2\varepsilon$-сеть для $A\times B \Rightarrow A\times B$ -- компакт
	
	Но для последнего следствия нужна еще и полнота $A\times B$, ну а это верно, тк $A$ и $B$ полные.
	
\end{proof}

\begin{theorem}\thmslashn
	
	$\mu X < +\infty$, $X$ и $T$ -- компакты, $f \in C(X\times T)$
	
	Тогда $F(t) = \int\limits_{X} f(x, t) \,d\mu (x) \in C(T)$
	
\end{theorem}

\begin{proof}\thmslashn
	
	$\abs{f(x, t)}\leqslant M \quad \forall x \in X \;\;\forall t \in T \quad \Phi(x):= M$ -- суммируемая мажоранта $\Rightarrow$ можем подставить в теорему $2.1$
	
\end{proof}

\begin{consequence}\thmslashn
	
	Если $\Omega \subset \R^n$ открыто, $\mu X < +\infty$, $X$ -- компакт
	
	$f\in C(X\times\Omega)$, тогда $F \in C(\Omega)$
	
\end{consequence}

\begin{proof}\thmslashn
	
	Проверим непрерывность $F$ в точке $t_0 \in \Omega$
	
	Тогда $\bar{B_r}(t_0) \subset \Omega$ для некоторого $r > 0$
	
	$f\in C(X \times \bar{B_r}(t_0)) \Rightarrow F \in C(\bar{B_r}(t_0)) \Rightarrow F\in C(B_r(t_0)) \Rightarrow F$ непрерывна в точке $t_0$.
	
\end{proof}